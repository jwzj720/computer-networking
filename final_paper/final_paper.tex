\documentclass[10pt, letterpaper, twocolumn]{article}
\usepackage[margin=1in]{geometry}

% Use CC style guide fonts rather than default fonts
\usepackage{fontspec}
\setmainfont[Ligatures=TeX]{Crimson}
\setsansfont[Ligatures=TeX]{Montserrat}

\begin{document}

\twocolumn[ \centerline{\Huge 
%%%% Title
Final Paper Template 
}
\vspace{1cm}
\centerline{
    
%%%% Team of 3 authors block
\begin{tabular}{ccc}
Alex the Bear     & Laurie Sloth        & Danielle Ellsworth            \\
alex@dellswor.net & laurie@dellswor.net & dellsworth@coloradocollege.edu\\
\end{tabular}

%%%% Team of 4 authors block
%\begin{tabular}{cc}
%Alex the Bear                  & Laurie Sloth        \\
%alex@dellswor.net              & laurie@dellswor.net \\
%\\
%Danielle Ellsworth             & Leapord             \\
%dellsworth@coloradocollege.edu & \\
%\end{tabular}

\vspace{1cm}
} ]

%% General Notes:
%%
%% Please avoid including a link to your git repo in thepaper. With your 
%% group's permission, I would like to share these papers with future 
%% deliveries of this class.
%%
%% The paper length (less citations and appendicies) is capped at not more
%% than 10 pages. A typical conference paper length is 6 to 8 pages. There's
%% not space to say everything, so you'll need to be a little choosy about
%% what goes in the paper and what doesn't. STEM papers tend to have a much 
%% more staccato writing style than humanities works, in part, due to the 
%% tension between the amount of content to present and the short page length
%% to present in.

\begin{abstract}
%% In a short paragraph or two, how would you summarize what a reader will
%% learn from reading your paper.
\end{abstract}

\section{Introduction}
%% Ground the reader in what this paper is about. They come from your
%% pre-CP341 understanding of computer networks... They won't know that 
%% signalling is a problem, or addressing, or the kind of data network
%% applications send. The details will be in the later sections, but we need
%% something to invite the reader to start thinking about these problems.


\section{Approach}
%% Like your progress through the project for this class, real research is a
%% terrible twisted mess of false starts and deadends... Any paper/book that
%% lays out a structured path through the research is a lie invented only after
%% the research is done. It's a narrative device that helps the reader connect
%% and order the ideas in the rest of the paper/book.

%% This approach section in this paper is intended to serve this narrative
%% purpose. You'll probably want to introduce a layered networking model and 
%% the role of each layer. You'll probably also want to outline the really 
%% high level details around how your layers interface with one another.
%% You'll likely want to introduce the hardware and project goals.


\section{Link Layer}
%% How does your link layer work? You shouldn't show code. You should talk
%% about your algorthims and how you found/solved certain problems. You 
%% probably need to talk about what the waves your writing/reading look like
%% and how that maps to data. You should be able to say things about what the 
%% limitations of your link layer are. This section needs to include bps 
%% you've measured and what kinds of error conditions are detected/handled.


\section{Network Layer}
%% How does your network layer work? You shouldn't show code. You should talk
%% about your algorthims and how you found/solved certain problems. You should 
%% be able to say things about what the limitations of your network layer are.
%% This section needs to talk about supported topologies, the structure of 
%% addresses, how the forwarding tables are maintained, how topology changes 
%% (e.g. a link coming up/going down) are handled). How much overhead does
%% your network layer add?


\section{Applications}
%% Talk about your applications in this section (and transport layer if you
%% implemented one). 

\subsection{Application 1}
%% What does your application do? What messages are needed? What is their
%% structure?

\subsection{Application 2}
%% What does your application do? What messages are needed? What is their
%% structure?


\section{Future Work}
%% If you had more time, what would you try to do/improve/understand next?


\section{Conclusion}
%% 1 or 2 paragraphs to summarize what was presented/learned.

\bibliographystyle{acm}
\bibliography{refs.bib}

\end{document}

