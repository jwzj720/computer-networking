\documentclass{article}
\usepackage[a4paper, margin=1in]{geometry}
\begin{document}

\section*{Progress Report 4}
Stuart, Walt, Dan

%% tasks and task owners for roughly the next 24 hours
\section*{Next Goals and Deliverables}
\begin{itemize}
\item 
\end{itemize}

%% Section enumerating progress from the previous day. Items people were responsible for in the last progress report must appear. Additional tasks that the group agreed to add since the last progress report may be presented here.
\section*{Previous Goals and Deliverables}
\begin{itemize}
    \item Debug program for receiving manchester encoding data (DONE) - Stuart/Walt
    \item Develop standard packet expectations - group (DONE for now -- will revist with network layer & with error correcting)
    \item Create character encoding/decoding to allow text transfer - (WIP) - Dan
        \subitem  Dan has made progress on this, but we needed to fix the physical link before doing this.
    \item develop functionality for repairing broken packets using error detection and bit replacement. (ON HOLD/DONE FOR NOW) - Dan
        \subitem Dan read a lot about this, and we have a good sense of where to start on error-detection. 
        However, we decided to move on to transmitting English messages before we implement this. 
\end{itemize}

%% Sections above should be short bullets. Longer free form discussion of information from above goes here. These should still be short! More that a couple sentences and the discussion is better suited as a results write-up, appendix, or design document.
\section*{Discussion}
\begin{itemize}
    \item We spent much of our class time trying to debug our read rate, only to realize there was nothing wrong with it. We realized
    that our bits were inverted due to the input/output nature of the NIC device, and some of our errors were being caused by an existing NIC state.
    \item We have a working program(s) for receiving manchester encoded data. We can send data at rates up to 3,000 bits per second. 
    This is a good start, and we are happy that we have reached the upper bound of what the class belives is possible for these devices.
    However, we will have to add error checking to our packets. Once we start to focus on the network layer, we will also need to come up with a 
    handshake for the devices to connect, and will need to add automatic timing resets to the listening function so the clocks don't get out of sync. 
    \item We use the 
    \item 
\end{itemize}

%% The appendix pages go here if there are any...
%\section*{Appendix}
\end{document}
