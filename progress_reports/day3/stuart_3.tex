\documentclass{article}
\usepackage[a4paper, margin=1in]{geometry}
\begin{document}

\section*{Progress Report 3}
Stuart, Walt, Dan

%% tasks and task owners for roughly the next 24 hours
\section*{Next Goals and Deliverables}
\begin{itemize}
\item Debug program for receiving manchester encoding data - Stuart
\item Develop standard packet expectations - group
\item Create character encoding/decoding to allow text transfer - Walt
\item develop functionality for repairing broken packets using error detection and bit replacement.
\end{itemize}

%% Section enumerating progress from the previous day. Items people were responsible for in the last progress report must appear. Additional tasks that the group agreed to add since the last progress report may be presented here.
\section*{Previous Goals and Deliverables}
\begin{itemize}
    \item Make bash script command for mounting USB on the Pi (DONE) - Walt
    \item Timing, error detection/correction research (in progress) - group
    \item Manchester encoding output program (DONE) - created by Walt (src/machester\_encoding\_send.c)
    \item Bash script for compiling and running files (DONE) - Walt
    \item Create program for reading the manchester encoding ouput(in progress) - Stuart
    \item Further research error detection and brainstorm bit correction functionality (in progress) - Dan
\end{itemize}

%% Sections above should be short bullets. Longer free form discussion of information from above goes here. These should still be short! More that a couple sentences and the discussion is better suited as a results write-up, appendix, or design document.
\section*{Discussion}
\begin{itemize}
\item We decided to move forward with manchester encoding because we felt that it would best solve our biggest anticipated issue: timing drift.
\item For our manchester coding reader, there seems to be an error with starting synchronization that is causing bits from the message to be the opposite of what is expected.
We're going to try and troubleshoot this in coming days to fix the consistency of the reading.
\item We were able to get a successful bit transmission, so now that we know how to communicate between two machines, an important
next step will be to build upon our bitwise communication. Our bits need to encode information, and as a group we will be deciding 
what our bits represent.
\item With our communication method perceived to be near completion, we can start broadening our scope to ideas like error detection to improve message accruacy.
\end{itemize}

%% The appendix pages go here if there are any...
%\section*{Appendix}
\end{document}
